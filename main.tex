\documentclass{article}
\setlength{\parskip}{5pt} % esp. entre parrafos
\setlength{\parindent}{0pt} % esp. al inicio de un parrafo
\usepackage{amsmath} % mates
\usepackage[sort&compress,numbers]{natbib} % referencias
\usepackage{url} % que las URLs se vean lindos
\usepackage[top=25mm,left=20mm,right=20mm,bottom=25mm]{geometry} % margenes
\usepackage{hyperref} % ligas de URLs
\usepackage{graphicx} % poner figuras
\usepackage[spanish]{babel} % otros idiomas
\usepackage[utf8]{inputenc}
\author{Elías Alejandro García Bueno \\
Bryan Alejandro Andrade Amaya} % author
\title{Tarea 1} % titulo
\date{26 de agosto de 2022}

\begin{document} % inicia contenido

\maketitle % cabecera

\begin{abstract} % resumen
La biomecánica es una ciencia de la rama de la bioingeniería y de la ingeniería biomédica, encargada del estudio, análisis y descripción del movimiento del cuerpo, además de examinar las fuerzas en función de la estructura biológica y los efectos producidos por esas fuerzas. 
\end{abstract}

\section{Introducción}\label{intro} % seccion y etiqueta

La biomecánica estudia lo seres vivos desde el punto de vista de la mecánica, buscando relaciones entre magnitudes y buscando explicaciones de comportamientos y observaciones. Dentro de la mecánica se incluye también todo lo relacionado con  los fluidos y la termodinámica.
En esta breve investigacion trataremos de explicar las bases de la biomecanica para entender mejor de que esta conformada.



%\\begin{figure} % figura
%    \centering
%    \includegraphics[width=150mm]{output3.jpg} % archivo
%    \caption{resultados del programa}
%    \label{grafica}
%\end{figure}

\section{Desarrollo}
\cite{aguilar2000biomecanica}\textbf{Biomecánica} \\

Se considerara en esta investigación como biomecánica a la disciplina que estudia los modelos, fenómenos y leyes que sean relevantes en el movimiento de un ser vivo. Para estudiar el movimiento hay que considerar tres aspectos distintos:

•	El control del movimiento que esta relacionado con los ámbitos psicológicos y neurofisiológico por lo que apenas se considerara en esta investigación.

•	La estructura del cuerpo que se mueve, que en el caso de los seres vivos es un sistema complejo compuesto por músculos, huesos, tendones, etc. Es la anatomía y la fisiología que aquí se estudiara desde un punto de vista mecánico.

•	Las fuerzas tanto externas (gravedad, viento, etc.) como internas (producidas por el propio ser vivo), que producen el movimiento de acuerdo con las leyes de la física.

Los dos últimos aspectos permiten el estudio de los movimientos de los seres vivos desde un punto de vista fundamentalmente anatómico o estructural. Así, los movimientos se deducen sobre todo de la estructura del sistema en movimiento aplicando tanto las leyes fisiológicas como físicas. Es la forma de ver a los seres vivos es lo que se conoce como kinesiología (teoría de los movimientos).

{\textbf{Magnitudes escalares y vectoriales}}

Se denominan cantidades escalares aquellas que sólo poseen un número que indica la cantidad y una unidad de medida. Así por ejemplo, 40 libras, 50 naranjas, 10 hombres, etc., son magnitudes escalares porque cumplen con las dos condiciones enumeradas.\\
Se denominan vectores a las cantidades que poseen cuatro condiciones: un número que indica cantidad, una unidad de medida, una dirección y un sentido. Así, la fuerza es una variable vectorial porque debe indicar las unidades de medida, la dirección y sentido en el cual se ejerce.
\vspace{5mm}


\item\cite{jose2009biomecanica}\textbf{Formas de movimiento} \

\textbf{Traslación}

La traslación (o movimiento linear) tiene lugar cuando un cuerpo4 mueve todas sus partes de manera que todas recorren el mismo espacio, en la misma dirección e intervalo de tiempo.
\vspace{5mm}

\textbf{Rotación}

El movimiento rotatorio (o movimiento angular) tiene lugar cuando todas las partes de un cuerpo se mueven a lo largo de una trayectoria circular alrededor de una línea (considerada como eje de rotación), con el mismo ángulo, al mismo tiempo. Este eje de rotación puede o no pasar por el cuerpo, pero siempre es perpendicular al plano de rotación
\vspace{5mm}

\textbf{Movimiento mixto o general}

Mientras la rotación es un movimiento más común que la traslación en las técnicas deportivas, lo es mucho más el movimiento mixto o general. Un ciclista que corre, por ejemplo, traslada su tronco en una trayectoria casi rectilínea, mientras que sus piernas realizan movimientos rotatorios.
\vspace{5mm}

\textbf{Cinemática lineal}

 La cinemática es la rama de la biomecánica que describe los movimientos sin tener en cuenta su causa. La cinemática lineal está relacionada con los movimientos de tipo lineal o curvilíneo.\\
 \vspace{5mm}
 
El movimiento se define como la variación de posición que experimenta un cuerpo en el transcurso del tiempo con respecto a un marco de referencia considerado como fijo. En el estudio de la cinemática se estudian las siguientes variables: 
\item

• Temporales: tiempo, frecuencia y período. \\
• Espaciales: distancia y desplazamiento. \\
• Espacio-temporales: velocidad, rapidez y aceleración. \\


\section{Conclusiones}
La biomecánica ha ayudado a la realización de prótesis para muchas personas las cuales por alguna razón clínica, accidentes, etc., no tienen una extremidad también es fundamental en el área de la terapia. Es importante conocer como funciona de manera física el cuerpo para poder evitar lesiones, como en el caso de la actividad física. El cuerpo tiene limites y se debe de saber cuales son para no ir más allá de ellos.


\bibliographystyle{plainnat}
\bibliography{bib}


\end{document}